\documentclass{article}
\usepackage{listings}
\usepackage{url}
\newcommand{\tw}[1]{\texttt{#1}}
\title{CppCMS --- Templates Library}
\author{Artyom Beilis (Tonkikh)}
\begin{document}
\maketitle
\tableofcontents
\lstlistoflistings
\begin{abstract}
This is full documentation of CppCMS templates library. This library was designed to work with CppCMS framework however
it can be used for any other purposes that even not connected to HTML generation.
\end{abstract}
\section{Key Components}
The templates system is build form several key components:
\begin{description}
\item[content] --- the object that holds all the information required for building the page.
\item[renderer] --- the object that holds a template and creates the page form the content
\item[template compiler] --- compiler that converts human readable template files to internal opcode that is used by renderer for generating pages.
\item[translation framework] --- the gettext compatible libraries that allows internationalize the templates content in multi-thread environment.
\end{description}

We would explain this in a simple example.
First of all we would create a text file \tw{test.tmpl} with the
code displayed in listing~\ref{lst:firsttmpl} and create a C++ code file \tw{test.cpp}
with the code in listing~\ref{lst:firstcpp}.

\begin{figure}
\begin{lstlisting}[label=lst:firsttmpl,caption=Frist Template]{}
<% template test() %>
Hello <% name %>!
<% end %>
\end{lstlisting}
\end{figure}


\begin{figure}
\begin{lstlisting}[label=lst:firstcpp,caption=Frist C++ Code]{}
#include <tmpl/content.h>
#include <tmpl/renderer.h>
#include <iostream>
using namespace tmpl;
using namespace std;

int main()
{
	template_data data("test.opcode");
	renderer rnd(data);
	content  cnt;
	string result;

	cnt["name"]=string("John");
	
	rnd.render(cnt,"test",result);
	cout<<result;
	return 0;
}
\end{lstlisting}
\end{figure}

Now we should build templates opcode by running:

\begin{verbatim}
tmpl_bld -o test.opcode test.tmpl
\end{verbatim}

Then compile our test code

\begin{verbatim}
g++ test.cpp -ltmpl -o test
\end{verbatim}

And now we can run it \tw{./test}. If everything was OK we should see as output:

\begin{quote}
Hello John!
\end{quote}

Now lets see line after line what happens. First we declare a template called `test', it includes some
text and special command \verb+<% name %>+ that displays the value of content variable `name'. When we run
\verb+tmpl_bld+ we create from text file \verb+test.tmpl+ a binary file \verb+test.opcode+ that actually 
includes a simple instructions for displaying the content. You can find an `assembly' code in file \verb+test.s+.

In the C++ code, we first of all load the opcode file we created to an object `data' of
type \verb+template_data+. It holds the content of the file and can be used by multiple 
`renderer' objects. After that we can create an object `rnd' of type \verb+renderer+ that would use our opcode.

The next step is setting up the content. We assign to a key \tw{name} value `John': 
the value that would be displayed for this content variable.
Now we are ready to render the output calling \verb+render+ method of renderer object.
It receives 3 parameters: the content, the name of the template and the string that
would be filled up with rendered content. And now we cat print it as usual.

\section{Content}

The \verb+tmpl::content+ class is actually derived from \verb+map<string,boost::any>+. It maps 
string key to a container that is capable of holding any\footnote{For full documentation 
of \tw{boost::any} refer to \url{http://www.boost.org/doc/html/any.html} } value.

In order to add a pair key/value to the content, you just assign it as if it was an ordinary map. 
Currently renderer supports \tw{int}, \tw{string}, \tw{bool} and \tw{std::tm} types that it is capable to display.
However, it can be extended to any user defined type, see~\ref{ref:NewTypes}.

The templates system is capable of representing sequence of data---iterate over them. For example see listing~\ref{lst:iter}.

\begin{figure}
\begin{lstlisting}[label=lst:iter,caption=Simple Iterating Over Sequences]{}
<% foreach person in people %>
<ul>
	<% item %>
	<li><% person.name %> is 
		<% person.age %> year(s) old.</li>
	<% end %> 
</ul>
<% end %>
\end{lstlisting}
\end{figure}

In order to make a sequence of data we assign to a key `people' one of the following types:
\begin{enumerate}
\item \verb+vector_t+ of type \verb+vector<content>+
\item \verb+list_t+ of type \verb+list<content>+
\end{enumerate}
It can be done using following methods:
\begin{enumerate}
\item \verb+vector_t &vector(string const &key, int size=0)+ 
 method creates a vector of contents of size \tw{size} at given \tw{key} and returns it;
\item \verb+list_t &list(string const &key)+ 
 method creates a list of contents at required key as well;
\end{enumerate}

You can see an example of preparing a content for template~\ref{lst:iter} at listing~\ref{lst:VectoAndList}
\begin{figure}
\begin{lstlisting}[label=lst:VectoAndList,caption=Creating a Sequence of Contents]{}
// Using Vector
content::vector_t &vec = 
	cnt.vector("people",people.size());
for{i=0;i<people.size();i++) {
	vec[i]["name"] = people[i].name;
	vec[i]["age"]  = people[i].age;
}

// Using List
content::list_t &lst = cnt.list("people");
for{i=0;i<people.size();i++) {
	lst.push_back(content());
	lst.back()["name"] = people[i].name;
	lst.back()["age"]  = people[i].age;
}\end{lstlisting}
\end{figure}

This approach is good for most of cases. However in some cases, in order to prevent 
a waste of memory resources, it is better to create
a data `on-the-fly' using callback functions. This can be achieved using
\tw{boost::signal} and \tw{boost::bind}\footnote{See: \url{http://www.boost.org/doc/html/signals.html}}.

The method that attaches a callback is defined as following:

\begin{verbatim}
typedef boost::signal<bool (content &)> callback_t;
void signal(string const &key,callback_t::slot_type slot);
\end{verbatim}

You actually should attach a function that receives as parameter and empty content and fills it up with data returning `true'. It should return `false' at the end of the sequence. See listing~\ref{lst:ContantCallback} for an example.

Take a note, at the beginning the value of \tw{i} is 0. When we get to the end of the sequence, the callback function returns
false and \emph{resets} the value of \tw{i} to 0 again. This is important in case, the template would try to iterate
over the same sequence twice. If the value will not be reset, then other iterations over the same sequence will be
treated as iteration over empty sequence.

\begin{figure}
\begin{lstlisting}[label=lst:ContantCallback,caption=Creating a Sequence With Callback]{}
bool sequence(content &c,int &i)
{
	if(i>=people.size()) {
		i=0;
		return false;
	}
	c["name"]=people[i].name;
	c["age"]=people[i].age;
	i++;
	return true;
}
...
void setup()
{
	int i=0;
	cnt.signal("people",boost::bind(sequence,_1,i));
}
\end{lstlisting}

\end{figure}


\end{document}
