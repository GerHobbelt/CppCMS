\documentclass[11pt]{article}

\usepackage{url}
\usepackage{hyperref}
\usepackage{html}
\hypersetup{colorlinks,%
citecolor=red,%
filecolor=red,%
linkcolor=red,%
urlcolor=green,%
pdftex}
\newcommand{\tw}[1]{\texttt{#1}}
\usepackage{float}

\floatstyle{ruled}
\newfloat{program}{thp}{lop}
\floatname{program}{Listing}

\begin{htmlonly}
	\newenvironment{\program}{
		\begin{rawhtml}
		<div class="figure" style="float:right; clear:right; width:500px; padding-left:5px;" >
		\end{rawhtml}
		}{
		\begin{rawhtml}
		</div>
		\end{rawhtml}
		}
	\renewcommand{\caption}[1]{\textbf{#1}}
\end{htmlonly}


\title{DBIxx User Manual}
\author{Artyom Tonkikh}
\begin{document}

\maketitle
\tableofcontents
%begin{latexonly}
\listof{program}{List of Listings}
%end{latexonly}

\begin{abstract}
DBIxx is a C++ wrapper for libdbi\footnote{\url{http://libdbi.sourceforge.net/}} --- a database-independent abstraction layer written in C. The major goal of this project is to provide an API useful for C++ programmers that support: exceptions, automatic creation and destruction of objects, integration with STL strings library and nice syntactic sugar preserving all the power of libdbi library.

This project was significantly influenced by the general ideas and syntactic sugar of soci\footnote{\url{http://soci.sourceforge.net/}} library, however it significantly differs from it in many design aspects. You may use any one of them -- according to your needs.
\end{abstract}
\section{Quick Start Guide}

A simple program can be found at listing~\ref{lst:quickstart}. As you can see, it is quite simple to understand what happens. But, lets go line after line and describe everything:

\begin{program}
\caption{Quick Start\label{lst:quickstart}}
\begin{verbatim}
#include <dbi/dbixx.h>            // Note 1
#include <iostream>              
using namespace dbixx;
using namespace std;

int main()
{
    try {
        session sql("mysql");    // Note 2

        sql.param("dbname","mydb");
        sql.param("username","alex");
        sql.param("password","secret");
                                 // Note 3

        sql.connect();           // Note 4
    
        sql<<"insert into test(id,str) values(?,?)",
            1,"Hello World";
                                 // Note 5
        sql.exec();              // Note 6

        row r;
        sql<<"select id,str from test where id=?",
            1;                   // Note 7
        sql.single(r);           // Note 8


        int id;
        string val;

        r >> id >> val;          // Note 9
        cout <<id << ':'<<val<<endl;
    }
    catch(dbixx_error &e) {      // Note 10
        cerr<<e.what()<<endl;
        return 1;
    }

    return 0;
}
\end{verbatim}
\end{program}


\begin{enumerate}
\item Include header and define the namespace the DBIxx wrapper uses.
\item Create a new session object that uses MySQL driver.
\item Bind all relevant parameters for this driver.
\item Create actual connection to the data base.
\item Set a query that will be executed and bind its two parameters. The parameters are marked with `?' symbol withing a text of the query, they are replaced with values `1' and ``Hello~World''.

All string parameters are automatically auto escaped.
\item Execute a query we had prepared.
\item Prepare a new query -- `selecting' a row;
\item Store the result to a `row' object.
\item Save values into actual variables: i, val;
\item Catch any kind of problems that can occur and display a error message.

All errors are thrown using \verb+dbixx::dbixx_error+ object that is derived from \verb+std::runtime_error+.
\end{enumerate}

\section{Interface}

DBIxx introduces four major classes:
\begin{itemize}
\item[\tw{session}] this class that is responsible on the connection with database. It uses us to produce queries, get result, establish connections etc.
\item[\tw{result}] this class represents a result of query -- similar to `\verb+dbi_row+'
\item[\tw{row}] this class is used to fetch data from single row.
\item[\tw{dbixx\_error}] this class used for throwing errors, it is derived from \verb+std::runtime_error+.
\item[\tw{transaction}] this is automatic transaction management class that allows you to manage transactions safely.
\end{itemize}



\end{document}
